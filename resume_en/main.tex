% !TeX TS-program = xelatex

\documentclass{resume}
\usepackage[style=ieee]{biblatex}

% Define the bibliography file
\addbibresource{own-bib.bib} % Replace "your_bibliography" with your actual BibTeX file name

\ResumeName{Frank Young}

\begin{document}

\ResumeContacts{
  % (+86) 131-6207-0162,%
  \ResumeUrl{mailto:2679frank@gmail.com}{2679frank@gmail.com},%
  % \ResumeUrl{https://blog.fkynjyq.com}{blog.fkynjyq.com} \footnote{Underlined content contains hyperlinks.},%
  \ResumeUrl{https://frank2679.github.io}{github.com/frank2679.io},%
  \ResumeUrl{https://www.linkedin.com/in/frank-young-7b5631288/}{Linkedin/FrankYoung}
}


\ResumeTitle


\section{Personal Summary}

\begin{itemize}
  \item Over six years of experience in heterogeneous platform (CPU/GPU/DSP/NPU) operator acceleration library, including DNN, BLAS, FFT, RAND. Deeply involved in 0-1 chip software and hardware development.
  \item Two years of team leadership experience, leading a team of 5 people.
  \item Strong learning ability, pursuit of excellence, and a passion for triathlons.
\end{itemize}

\section{Education}
\ResumeItem
[National Taiwan University|Master's Degree]
{National Taiwan University}
[\textnormal{Telecom Institute|} Master's Degree]
[2014.09—2017.01]

\nocite{*} % Include all publications from the BibTeX file
%% If you just want everything in one list
\printbibliography[heading={none}]

% \printbibliography[heading={subbibliography},title={Journal Articles},type=article]
% \printbibliography[heading={subbibliography},title={Conference Proceedings},type=inproceedings]


\ResumeItem
[Chongqing University|Undergraduate]
{Chongqing University}
[\textnormal{Communication Engineering|} Bachelor's Degree]
[2010.09—2014.06]

\textbf{Top 5\%} of the program, outstanding graduate, multiple scholarship recipient.

% \section[Technical Skills]{Technical Skills\protect\footnote{Skills unrelated to the job position are omitted or represented in gray.}}
\section{Technical Skills}

\begin{itemize}
  \item \textbf{Technology Stack}: Familiar with operator optimization, deep learning model deployment optimization, including CNN, large language models.
  \item \textbf{Tools}: Proficient in C++, Python, CUDA, CMake, Verdi, Vim, Git, JIRA, and working in an English environment.
\end{itemize}

\section{Work Experience/6 Years}

\ResumeItem{Leading GPU Manufacturer in China/3 Years}
[Software Engineer/Team Leader]
[2021.1—Present]

\begin{itemize}
  \item Developed AI/HPC acceleration libraries from scratch based on GPGPU platform.
  \item Led and managed a team of around 10 people, providing guidance and direction for project development.
\end{itemize}

\ResumeItem{Leading company of video surveillance products/3.5 Years}
[Software Engineer]
[2017.7—2020.12] 

\begin{itemize}
  \item Developed high-performance convolutional neural network (CNN) library for heterogeneous platforms, with a focus on evaluating chip performance.
  \item Held end-to-end responsibility for algorithm-side projects, deeply analyzing business requirements, and building efficient application solutions to accelerate intelligent algorithm implementation.
\end{itemize}

\section{Project Experience}

\ResumeItem{\textbf{BLAS} Library in C++}
[Development and Maintenance from Scratch]
[2021.11—Present]

\begin{itemize}
  \item Led the project, responsible for design, construction, development, testing, CICD, documentation, and management.
  \item Optimized the performance of various large language model GEMM scenarios, benchmarked against A100. 
  \item Designed and implemented kernel selection algorithms.
  \item Implemented operator registration framework, logging system, and core operator development.
\end{itemize}

\ResumeItem{Ultimate Optimization of GEMM}
[based on verdi/zebu]
[2022.9—2022.11]

\begin{itemize}
  \item Handwritten assembly to achieve ultimate performance GEMM on Vcore, achieved full utilization for specific shapes by observing waveforms through Verdi.
  \item Implemented simulated FP32 GEMM algorithm based on Tcore, achieved full utilization for specific shapes, and outperformed A100 in performance.
\end{itemize}

\ResumeItem{DNN Framework Development}
[Self-defined yaml to serialize operator and graph]
[2022.3—Present]

\begin{itemize}
  \item Developed a deep learning framework based on Caffe.
  \item Self-defined operator/graph expression based on YAML referencing ONNX.
  \item Compared precision with mainstream industry solutions and designed custom precision comparison solutions.
  \item Deployed precision verification, and performance dashboard.
\end{itemize}


\end{document}

