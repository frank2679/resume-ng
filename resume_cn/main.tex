% !TeX TS-program = xelatex

\documentclass{resume}
\usepackage[style=ieee]{biblatex}

% Define the bibliography file
\addbibresource{own-bib.bib} % Replace "your_bibliography" with your actual BibTeX file name

\ResumeName{杨行}

\begin{document}

\ResumeContacts{
  (+86) 151-6143-5537,%
  \ResumeUrl{mailto:2679frank@gmail.com}{2679frank@gmail.com},%
  % \ResumeUrl{https://blog.fkynjyq.com}{blog.fkynjyq.com} \footnote{下划线内容包含超链接。},%
  \ResumeUrl{https://frank2679.github.io}{github.com/frank2679.io},%
}


\ResumeTitle·

\section{个人总结}
\begin{itemize}
    \item 拥有7年异构平台(CPU/GPU/DSP/NPU)算子加速库开发经验,专注 DNN、BLAS 等核心加速库的开发。
    \item 具有三年团队管理经验,成功带领团队完成交付,得到客户认可,也曾跨团队主导项目开发。
    \item 拥有超强的学习能力,追求卓越,喜欢钻研技术,乐于分享。
\end{itemize}

% \section[技术能力]{技术能力\protect\footnote{与求职岗位无关的技能省略或用灰色表示。}}
\section{技术能力}
\begin{itemize}
    \item \textbf{技术方向}: 加速库开发、算子优化、深度学习模型部署优化(LLM,CNN),熟悉 pytorch,cutlass。
    \item \textbf{编程语言}: C++/C、Python、CUDA。
    \item \textbf{开发工具}: CMake、Verdi、Vim、Git、JIRA、Jenkins、markdown。
\end{itemize}


\section{工作经历}

\ResumeItem{国内头部GPU厂商} 
[软件工程师 / Team Leader | 算子加速库]
[2021.01 至今]
\begin{itemize}
    \item 从零开始开发AI领域的加速库(DNN、BLAS),在多种大模型场景下优化性能,达到A100同等水平。
    \item 设计并实现了高效的算子注册框架和日志系统,以及kernel selection算法,大幅提升计算效率。
    \item 负责项目从设计、开发、测试到CICD的全流程管理,确保高质量交付。
\end{itemize}

\ResumeItem{安防领域龙头企业} 
[软件工程师 | AI加速库 / 商业应用]
[2017.07 — 2020.12]
\begin{itemize}
    \item 负责异构平台高性能卷积神经网络(CNN)库的开发,优化芯片性能,提升算法落地效率。
    \item 从业务需求到解决方案的端到端项目负责,构建高效的应用方案,实现智能算法的实际应用。
\end{itemize}

\section{项目经历}

\ResumeItem{BLAS 支持 llama2 系列} [llama2 性能打平 A100 | 专利 2 项]
[2023.12 - 2024.03]
\begin{itemize}
    \item 通过 trace 分析llama2系列模型不同并行策略下 BLAS 所需要优化的各种场景。
    \item 使能了多种优化策略,包括 warp-specialized 编程范式, fused bias, fused grad 累加,BF16 累加等。
\end{itemize}

\ResumeItem{GEMM极致优化实现}[芯片利用率打满 | 专利 2 项]
[2021.09 - 2021.10]
\begin{itemize}
    \item 在特定shape下,通过手写汇编和Verdi波形分析,实现了基于vcore和tcore平台的GEMM算法性能最优化。
\end{itemize}

\ResumeItem{DNN框架开发}[支撑DNN 全栈验证|软著 1项,软著1项]
[2022.03 - 2023.06]
\begin{itemize}
    \item 参考pytorch开发深度学习框架,并设计基于YAML的算子表达方式,实现了自动化精度验证和性能对标。
\end{itemize}

\section{教育经历/硕士研究生}
\ResumeItem
[台湾大学|硕士研究生]
{台湾大学}
[\textnormal{电信研究所|}  学术型硕士研究生]
[2014.09 —2017.01]
WIFI 802.11ax 协议设计,分别在 IEEE access 期刊,及通信领域顶会 globalcom 上发表两篇论文。

\nocite{*} % Include all publications from the BibTeX file
%% If you just want everything in one list

% \printbibliography[heading={subbibliography},title={Journal Articles},type=article]
% \printbibliography[heading={subbibliography},title={Conference Proceedings},type=inproceedings]


\ResumeItem
[重庆大学|本科生]
{重庆大学}
[\textnormal{通信学院|} 工学学士]
[2010.09 —2014.06]

\textbf{专业前 5\%},优秀毕业生,获奖学金多次

\section{其他信息}
\begin{itemize}
    \item \textbf{兴趣爱好}: 铁三爱好者,持续学习新技术,积极参与技术社区分享。
    \item AI 领域专利 5 项,软著 1 项,通信领域专利一个,论文2篇。
    \printbibliography[heading={none}]
\end{itemize}


\end{document}
