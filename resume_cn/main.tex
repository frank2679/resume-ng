% !TeX TS-program = xelatex

\documentclass{resume}
\usepackage[style=ieee]{biblatex}

% Define the bibliography file
\addbibresource{own-bib.bib} % Replace "your_bibliography" with your actual BibTeX file name

\ResumeName{黄杨}

\begin{document}

\ResumeContacts{
  % (+86) 131-6207-0162,%
  \ResumeUrl{mailto:2679frank@gmail.com}{2679frank@gmail.com},%
  % \ResumeUrl{https://blog.fkynjyq.com}{blog.fkynjyq.com} \footnote{下划线内容包含超链接。},%
  \ResumeUrl{https://frank2679.github.io}{github.com/frank2679.io},%
  \ResumeUrl{https://www.linkedin.com/in/frank-young-7b5631288/}{Linkedin/FrankYoung}
}


\ResumeTitle


\section{个人总结}

\begin{itemize}
  \item 有六年以上异构平台(CPU/GPU/DSP/NPU)算子加速库经验,包括 DNN, BLAS, FFT 等。深度参与0-1芯片软硬件研发;
  \item 有两年带团队经验,团队规模5人;
  \item 超强的学习能力,追求卓越,铁三爱好者。
\end{itemize}

\section{教育经历/硕士研究生}
\ResumeItem
[台湾大学|硕士研究生]
{台湾大学}
[\textnormal{电信研究所|}  学术型硕士研究生]
[2014.09—2017.01]
WIFI 802.11ax 协议设计,分别在 IEEE access 期刊,及通信领域顶会 globalcom 上发表两篇论文。

\nocite{*} % Include all publications from the BibTeX file
%% If you just want everything in one list
\printbibliography[heading={none}]

% \printbibliography[heading={subbibliography},title={Journal Articles},type=article]
% \printbibliography[heading={subbibliography},title={Conference Proceedings},type=inproceedings]


\ResumeItem
[重庆大学|本科生]
{重庆大学}
[\textnormal{通信学院|} 工学学士]
[2010.09—2014.06]

\textbf{专业前 5\%},优秀毕业生,获奖学金多次

% \section[技术能力]{技术能力\protect\footnote{与求职岗位无关的技能省略或用灰色表示。}}
\section{技术能力}

\begin{itemize}
  \item \textbf{技术内容}: 熟悉算子优化,深度学习模型部署优化,包括 CNN, 大语言模型;
  \item \textbf{技术栈}: 熟悉 C++, Python, cuda, cmake, verdi, vim, git, JIRA, 英文工作环境.
\end{itemize}

\section{工作经历/6年}

\ResumeItem{国内头部GPU厂商/3年}
[软件工程师/team leader/算子加速库]
[2021.1 至今]

\begin{itemize}
  \item 基于 GPGPU 平台从零开发 AI/HPC 领域加速库(DNN, BLAS, RAND, FFT)
  \item 带领5人团队
\end{itemize}

\ResumeItem{安防领域龙头企业/3年半}
[软件工程师/AI 加速库/商业应用]
[2017.7—2020.12] 

\begin{itemize}
  \item \textbf{异构平台高性能卷积神经网络CNN库}开发,调研评估芯片性能;
  \item 端到端负责算法侧项目,深入梳理业务需求,构建高效的应用方案,加速智能算法落地;
\end{itemize}

\section{项目经历}

\ResumeItem{\textbf{BLAS} 加速库}
[0-1开发维护]
[2021.11-至今]
\begin{itemize}
  \item lead 项目,负责设计,构建,开发,测试,CICD,文档,管理;
  \item GEMM 优化,优化多种大模型场景,性能对标 A100。
  \item kernel selection 算法设计与实现;
  \item 实现了算子注册框架,日志系统,核心算子开发,
\end{itemize}

\ResumeItem{\textbf{GEMM} 极致优化实现}
[基于 zebu/verdi 极致优化]
[2021.9—2021.10]

\begin{itemize}
  \item 手写汇编在 vcore 上实现极致性能 GEMM,通过 verdi 看波形,做到特定 shape 下利用率打满。
  \item 基于 tcore 实现模拟 fp32 GEMM 算法,也做到特定 shape 下利用率打满,性能超过A100。
\end{itemize}

\ResumeItem{开发 \textbf{DNN} 框架}
[基于 yaml 自定义算子表达]
[2022.3—2023.6]

\begin{itemize}
  \item 参考 caffe 开发深度学习框架;
  \item 参考 onnx 设计基于 yaml 定义算子/图表达;
  \item 对标业界主流精度对比方案,自定义精度比对方案;
  \item 自动化部署精度验证,benchmark dashboard;
\end{itemize}


\end{document}
